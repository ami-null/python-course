% \documentclass[12pt, handout]{beamer}
\documentclass[12pt, aspectratio=169]{beamer}

\usetheme{moloch}
\molochset{block=fill}

\usepackage{fontspec}
\setsansfont[
    UprightFont = Inter-Light,          % Use light as the normal font weight
    BoldFont = Inter-SemiBold,           % Use semibold for \textbf
    ItalicFont = Inter-LightItalic,      % Light italic for \textit
    BoldItalicFont = Inter-SemiBoldItalic % Semibold italic for \textbf with \textit
]{Inter}[RawFeature={+ss04, +ss03, +dlig, +tnum}]
% Inter font stylistic sets:
% ss01: alternate digits for 3, 4, 6, 8
% ss02: for disambiguation (with zero) in places like "Ill" or "O0" etc
% ss04: for disambiguation (without zero) in places like "Ill" etc
% ss03: round comma, quation marks

\usepackage{sourcecodepro}

\usepackage{upquote}
\usepackage{microtype}
\UseMicrotypeSet[protrusion]{basicmath}    % disable protrusion for tt fonts

\usepackage{amsmath}
\usepackage{amssymb}
\usepackage{unicode-math}
\setmathfont{Erewhon Math}[Scale=1.14]

\usepackage{hyperref}
\pdfstringdefDisableCommands{\def\translate#1{#1}}
\usepackage{bookmark}
\usepackage{url}

\usepackage{natbib}
\usepackage{appendixnumberbeamer}
\usepackage{enumerate}
% \usepackage{enumitem}    % it is giving the error: TeX capacity exceeded
% \usepackage{footnotehyper}
\usepackage{graphicx}
\usepackage{caption}
% \usepackage{subcaption}
\usepackage{booktabs}
\usepackage{makecell}
\usepackage{array}
\newcolumntype{H}{>{\setbox0=\hbox\bgroup\let\pm\relax}c<{\egroup}@{}}
% \newcolumntype{H}{>{\setbox0=\hbox\bgroup}c<{\egroup}}% <--- removed @{}
% https://tex.stackexchange.com/questions/567724/can-i-hide-a-table-column-with-the-s-type-from-siunitx
% https://tex.stackexchange.com/questions/414143/hide-column-without-adding-whitespace-to-table

\usepackage{minted}
\renewcommand{\theFancyVerbLine}{\ttfamily{\small\oldstylenums{\arabic{FancyVerbLine}}}}

\definecolor{airforceblue}{rgb}{0.36, 0.54, 0.66}
\hypersetup{
    colorlinks=true,
    linkcolor={mDarkTeal},    % this colour is defined by the moloch theme
    filecolor={Maroon},
    citecolor={airforceblue!120},
    urlcolor={airforceblue},
    pdfcreator={xelatex},
    bookmarksopen=true,    % Expand bookmarks in the PDF
    bookmarksnumbered=true % Include numbering in bookmarks
}

\bibliographystyle{apalike}

\let\oldcite=\cite
\renewcommand{\cite}[1]{\textcolor{airforceblue!120}{\oldcite{#1}}}
\let\oldcitet=\citet
\renewcommand{\citet}[1]{\textcolor{airforceblue!120}{\oldcitet{#1}}}
\let\oldcitep=\citep
\renewcommand{\citep}[1]{\textcolor{airforceblue!120}{\oldcitep{#1}}}


% \setlength{\leftmargini}{0em}
\setbeamercolor{page number in head/foot}{fg=gray}
\setbeamertemplate{footline}[frame number]
%\setbeamertemplate{itemize items}[circle]
\setbeamertemplate{enumerate items}[circle]
\setbeamertemplate{sections/subsections in toc}[circle]
\setbeamertemplate{frametitle continuation}[from second][(cont.)]
\setbeamercovered{transparent}
\beamertemplatenavigationsymbolsempty


\AtBeginSubsection[]{
    {
        \begin{frame}[noframenumbering, plain]
            \subsectionpage
        \end{frame}
    }
}


\newcommand\Wider[2][4em]{%
    \makebox[\linewidth][c]{%
        \begin{minipage}{\dimexpr\textwidth+#1\relax}
            % \raggedright#2
            \centering#2
        \end{minipage}%
    }%
}

% \newenvironment{myitemize}{
%     \begin{itemize}
%         \vspace{1em}
%         \setlength{\itemsep}{0.7\baselineskip}
% }{
%         \vspace{1em}
%     \end{itemize}
% }

% \newenvironment{myenumerate}{
%     \begin{enumerate}
%         \vspace{1em}
%         \setlength{\itemsep}{0.7\baselineskip}
% }{
%         \vspace{1em}
%     \end{enumerate}
% }


\title{Introduction to Python}
\author{Md. Aminul Islam Shazid}
\date{}


\begin{document}
    {
		\setbeamertemplate{footline}{}    % NO FOOTLINE FOR THESE TWO FRAMES
		\addtocounter{framenumber}{-2}    % not counting the title page and the outline in frame numbers

		\begin{frame}
			\titlepage
		\end{frame}

		\begin{frame}{Outline}
            \vfill
			\tableofcontents[subsectionstyle=hide]
            \vfill
		\end{frame}
	}

    \section{Introduction and History}

    \begin{frame}{What is Python?}
        \begin{itemize}
            \item High-level, interpreted, general-purpose programming language
            \item Emphasizes readability and simplicity
            \item Python philosophy: "Readability counts"
        \end{itemize}
    \end{frame}


    \begin{frame}{History of Python}
        \begin{itemize}
            \item Created by Guido van Rossum in 1991
            \item Influences: ABC, Modula-3, Unix shell scripting
            \item Designed for scripting, prototyping, and general programming
        \end{itemize}
    \end{frame}


    \begin{frame}{Original Goals and Evolution}
        \begin{itemize}
            \item Original goals: ease of use, readability, ``batteries included"
            \item Evolution:
            \begin{itemize}
                \item From scripting to web development, data science, AI
                \item Strong ecosystem and community growth
            \end{itemize}
        \end{itemize}
    \end{frame}


    \begin{frame}{The Zen of Python}
        Key principles:
        \begin{itemize}
            \item Beautiful is better than ugly
            \item Simple is better than complex
            \item Readability counts
            \item There should be one, and preferably only one, obvious way to do it
        \end{itemize}

        Run \texttt{import this} to view all of it.
    \end{frame}


    \section{Language Features}

        \begin{frame}{Basic Language Features}
        \begin{itemize}
            \item Interpreted, dynamically typed
            \item High-level, cross-platform
            \item Large standard library
        \end{itemize}
    \end{frame}


    \begin{frame}{Interpreters and Implementations}
        \begin{itemize}
            \item CPython (default, reference implementation)
            \item PyPy (JIT compilation)
            \item Cython, Numba (performance optimization)
            \item MicroPython (embedded)
            \item Jython (Java), IronPython (.NET)
            \item Pyodide (CPython ported to WebAssembly), Brython (Python interpreter written in JavaScript)
        \end{itemize}
    \end{frame}


    \begin{frame}{Interpreted vs Compiled}
        \begin{itemize}
            \item Interpreted: executes directly, easier debugging, slower
            \item Compiled: precompiled to machine code, faster, less flexible
            \item While Python is primarily an interpreted language, one can use both approaches via JIT or AOT compilers
        \end{itemize}
    \end{frame}


    \begin{frame}{High-Performance Compilers and Tools}
        \begin{itemize}
            \item PyPy, Cython, Numba for speed
            \item GPU-accelerated workloads supported via Polars, cuDF, OneAPI, ROCm
            \item Distributed computing via Dask, Ray, PySpark, Modin
        \end{itemize}
    \end{frame}


    \begin{frame}{Interoperability with Other Languages}
        \begin{itemize}
            \item C, C++, Fortran extensions
            \item Java (via Jython), R (via rpy2)
            \item Integration for high-performance or legacy code
        \end{itemize}
    \end{frame}


    \begin{frame}{Object-Oriented Programming}
        \begin{itemize}
            \item Classes, inheritance, polymorphism
            \item Encapsulation of data and behavior
            \item Widely used in web frameworks, GUI apps, simulations
        \end{itemize}
    \end{frame}


    \begin{frame}{Functional Programming}
        \begin{itemize}
            \item Functions are treated as first-class objects
            \item map(), filter(), reduce() and other higher order functions
            \item Comprehensions and generators
        \end{itemize}
    \end{frame}


    \section{Why Learn Python?}

    \begin{frame}{Why Learn Python}
        \begin{itemize}
            \item Simplicity \& readability → low entry barrier
            \item Versatility:
            \begin{itemize}
                \item Web development: Django, Flask
                \item Backend \& APIs: FastAPI, Flask
                \item Desktop applications: GUI apps, utilities
                \item Mobile development: Kivy, BeeWare (smaller adoption)
                \item Automation \& scripting: one-off scripts, data cleaning, DevOps
            \end{itemize}
            \item Community \& ecosystem → global adoption, extensive libraries
        \end{itemize}
    \end{frame}


    \begin{frame}{Python in Action — Famous Apps \& Services}
        \begin{itemize}
            \item Web: Instagram, Pinterest, Reddit, Spotify, Netflix, YouTube, Quora
            \item Desktop: Dropbox, BitTorrent, Calibre, Anki, Blender (Python scripting)
        \end{itemize}
    \end{frame}


    \begin{frame}{Why Learn Python for Data Science?}
        \begin{itemize}
            \item Core libraries: NumPy, Pandas, Polars
            \item ML/AI frameworks: Scikit-learn, TensorFlow, PyTorch
            \item Visualization: Matplotlib, Seaborn, Plotly
            \item Big Data \& cloud: PySpark, Dask
            \item Dominant language for AI research and production
        \end{itemize}
    \end{frame}


    \begin{frame}{PyPI and the Library Ecosystem}
        \begin{itemize}
            \item Over 500k community made packages
            \item Domains: web dev, scientific computing, ML/AI, automation
            \item Easy to install and manage via \texttt{pip} and other package managers
        \end{itemize}
    \end{frame}


    \begin{frame}{Solving the Two-Language Problem}
        \begin{itemize}
            \item Rapid prototyping in Python
            \item Production optimization via Cython, Numba, Rust, or C++ integration
            \item Combines speed of compiled languages with ease of Python
        \end{itemize}
    \end{frame}


    \begin{frame}{Current State of Python}
        \begin{itemize}
            \item Widespread adoption: academia, enterprise, startups
            \item Strong ecosystem of libraries and tools
            \item Significant performance improvements in Python 3.13 as well as introduction of an experimental Just-In-Time (JIT) compiler
        \end{itemize}
    \end{frame}


    \begin{frame}{Future Prospects and Innovation}
        \begin{itemize}
            \item Continued performance improvements (Python 3.14+)
            \item Wider adoption of GPU acceleration and distributed computing
            \item Rust integration, WebAssembly, cloud-native tools
            \item Expansion into embedded and edge devices
            \item Growth in AI, data science, scientific computing, and high-performance applications
        \end{itemize}
    \end{frame}


    \begin{frame}{Python vs Other Languages}
        \begin{itemize}
            \item Python vs C: simplicity vs raw performance
            \item Python vs R: general-purpose vs domain-specific (statistics)
            \item Complementary use cases; Python bridges prototyping \& production
        \end{itemize}
    \end{frame}


    \section{Getting Started with Python}

    \begin{frame}{Getting Started with Python}
        \begin{itemize}
            \item Easiest option: coding on Kaggle or Google Colab
            \item To code on local machine:
            \begin{itemize}
                \item For Windows and MacOS, download Python from \href{https://www.python.org/}{python.org}
                \item On most Linux distributions, Python is already installed
                \item Once Python is installed, need an IDE
                \begin{itemize}
                    \item Recommended: VSCode with Python extension
                    \item Other options include: PyCharm, Spyder
                \end{itemize}
            \end{itemize}

            Another easy option is to code using \href{https://thonny.org/}{Thonny}, it is an IDE purpose built for begginers learning Python, comes with Python bundled.
        \end{itemize}
    \end{frame}

\end{document}
